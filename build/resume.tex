%-------------------------------------
% LaTeX Resume for Software Engineers
% Author : Leslie Cheng
% License : MIT
%-------------------------------------

\documentclass[letterpaper,12pt]{article}[leftmargin=*]

\usepackage[empty]{fullpage}
\usepackage{enumitem}
\usepackage{ifxetex}
\ifxetex
  \usepackage{fontspec}
  \usepackage[xetex]{hyperref}
\else
  \usepackage[utf8]{inputenc}
  \usepackage[T1]{fontenc}
  \usepackage[pdftex]{hyperref}
\fi
\usepackage{fontawesome}
\usepackage{PlayfairDisplay}
\usepackage{xcolor}
\usepackage{tabularx}

%-------------------------------------------------- SETTINGS HERE --------------------------------------------------
% Header settings
\def \fullname {D. Yiğit Yılmaz}
\def \subtitle {\textbf{Software Engineer}}

\def \linkedinicon {\faLinkedin}
\def \linkedinlink {https://linkedin.com/in/d-yigit-yilmaz/}
\def \linkedintext {/d-yigit-yilmaz}

\def \phoneicon {\faPhone}
\def \phonetext {+31 6 30 146 510}

\def \emailicon {\faEnvelope}
\def \emaillink {mailto:doganyigityilmaz@gmail.com}
\def \emailtext {doganyigityilmaz@gmail.com}

\def \githubicon {\faGithub}
\def \githublink {https://github.com/yvesyil}
\def \githubtext {/yvesyil}

\def \websiteicon {\faGlobe}
\def \websitelink {https://yvesyil.xyz}
\def \websitetext {yvesyil.xyz}

\def \pronounciationlink {http://ipa-reader.xyz/?text=ji\%CB\%88it\%20j\%C9\%AF\%C9\%AB\%CB\%88maz&voice=Filiz}

\def \headertype {\doublecol} % \singlecol or \doublecol

% Misc settings
\def \entryspacing {-0pt}

\def \bulletstyle {\faAngleRight}

% Define colours
\definecolor{primary}{HTML}{000000}
\definecolor{secondary}{HTML}{FA4C11}
\definecolor{accent}{HTML}{263238}
\definecolor{links}{HTML}{FC6C05}

%------------------------------------------------------------------------------------------------------------------- 

% Defines to make listing easier
\def \linkedin {\linkedinicon \hspace{3pt}\href{\linkedinlink}{\linkedintext}}
\def \phone {\phoneicon \hspace{3pt}{ \phonetext}}
\def \email {\emailicon \hspace{3pt}\href{\emaillink}{\emailtext}}
\def \github {\githubicon \hspace{3pt}\href{\githublink}{\githubtext}}
\def \website {\websiteicon \hspace{3pt}\href{\websitelink}{\websitetext}}
\def \howtopronounce {\faInfoCircle \hspace{3pt}\href{\pronounciationlink}{How to pronounce my name}}

% Adjust margins
\addtolength{\oddsidemargin}{-0.55in}
\addtolength{\evensidemargin}{-0.55in}
\addtolength{\textwidth}{1.1in}
%\addtolength{\topmargin}{-0.6in}
\addtolength{\topmargin}{-0.2in}
\addtolength{\textheight}{1.1in}

% Define the link colours
\hypersetup{
    colorlinks=true,
    urlcolor=links,
}

% Set the margin alignment 
\raggedbottom
\raggedright
\setlength{\tabcolsep}{0in}

%-------------------------
% Custom commands

% Sections
\renewcommand{\section}[2]{\vspace{5pt}
  \colorbox{secondary}{\color{white}\raggedbottom\normalsize\textbf{{#1}{\hspace{7pt}#2}}}
}

% Entry start and end, for spacing
\newcommand{\resumeEntryStart}{\begin{itemize}[leftmargin=2.5mm]}
\newcommand{\resumeEntryEnd}{\end{itemize}\vspace{\entryspacing}}

% Itemized list for the bullet points under an entry, if necessary
\newcommand{\resumeItemListStart}{\begin{itemize}[leftmargin=4.5mm]}
\newcommand{\resumeItemListEnd}{\end{itemize}}

% Resume item
\renewcommand{\labelitemii}{\bulletstyle}
\newcommand{\resumeItem}[1]{
  \item\small{
    {#1 \vspace{-2pt}}
  }
}

% Entry with title, subheading, date(s), and location
\newcommand{\resumeEntryTSDL}[4]{
  \vspace{-1pt}\item[]
    \begin{tabularx}{0.97\textwidth}{X@{\hspace{60pt}}r}
      \textbf{\color{primary}#1} & {\color{accent}\small#2} \\
      \textit{\color{accent}\small#3} & \textit{\color{accent}\small#4} \\
    \end{tabularx}\vspace{-6pt}
}

% Entry with title and date(s)
\newcommand{\resumeEntryTD}[2]{
  \vspace{-1pt}\item[]
    \begin{tabularx}{0.97\textwidth}{X@{\hspace{60pt}}r}
      \textbf{\color{primary}#1} & {\color{accent}\small#2} \\
    \end{tabularx}\vspace{-6pt}
}

% Entry for special (skills)
\newcommand{\resumeEntryS}[2]{
  \item[]\small{
    \textbf{\color{primary}#1 }{ #2 \vspace{-6pt}}
  }
}

% Double column header
\newcommand{\doublecol}[6]{
  \begin{tabularx}{\textwidth}{Xr}
    {
      \begin{tabular}[c]{l}
        \fontsize{25}{35}\selectfont{\color{primary}{{\textbf{\fullname}}}} \\
        {\textit{\subtitle}} % You could add a subtitle here
      \end{tabular}
    } & {
      \begin{tabular}[c]{l@{\hspace{1.5em}}l}
        {\small#4} & {\small#1} \\
        {\small#5} & {\small#2} \\
        {\small#6} & {\small#3}
      \end{tabular}
    }
  \end{tabularx}
}

% Single column header
\newcommand{\singlecol}[6]{
  \begin{tabularx}{\textwidth}{Xr}
    {
      \begin{tabular}[b]{l}
        \fontsize{35}{45}\selectfont{\color{primary}{{\textbf{\fullname}}}} \\
        {\textit{\subtitle}} % You could add a subtitle here
      \end{tabular}
    } & {
      \begin{tabular}[c]{l}
        {\small#1} \\
        {\small#2} \\
        {\small#3} \\
        {\small#4} \\
        {\small#5} \\
        {\small#6}
      \end{tabular}
    }
  \end{tabularx}
}

\begin{document}
%-------------------------------------------------- BEGIN HERE --------------------------------------------------

%---------------------------------------------------- HEADER ----------------------------------------------------

\headertype{\linkedin}{\github}{\website}{\phone}{\email}{\howtopronounce} % Set the order of items here

\vspace{2pt}

\textit{Amsterdam, Netherlands}

\vspace{15pt}

The quote "What I cannot create, I do not understand" by Richard Feynman really resonates with me because this is my approach for learning new concepts in life. If I find myself trying to understand how a system works, I start by thinking how I would design it and build a small version of it.

\vspace{15pt}

%-------------------------------------------------- EDUCATION --------------------------------------------------
\section{\faGraduationCap}{Education}

  \resumeEntryStart
    \resumeEntryTSDL
      {Technological University of the Shannon (TUS)}{2021 -- 2023}
      {B.Sc Honours in Honours Software Engineering with Cloud Computing}{Athlone, Ireland}
    \resumeItemListStart
      \resumeItem {Grade: First Class, Courses: Distributed Systems, Databases, Service Oriented Architecture, Security}
    \resumeItemListEnd
    \resumeEntryTSDL
      {Bilkent University}{2019 -- 2021 (Transferred to TUS)}
      {B.Sc in Information Systems and Technologies}{Ankara, Turkey}
    \resumeItemListStart
      \resumeItem {Grade: 3.2/4.0, Courses: Object Oriented Analysis and Design, Computer Algorithms and Data Structures, Computer Networks Cisco CCNA, Web Technologies}
    \resumeItemListEnd
  \resumeEntryEnd

%-------------------------------------------------- EXPERIENCE --------------------------------------------------
\section{\faPieChart}{Experience}

  \resumeEntryStart
    \resumeEntryTSDL
      {ING Bank Nederland}{October 2023 -- Present}
      {Software Developer}{Amsterdam, Netherlands}
    \resumeItemListStart
      \resumeItem {Currently involved on the improvement and the migration of billing services to onboard business customers to a new platform within daily banking.}
      \resumeItem {Developed test tools using Python and Selenium that automated the regression testing of the migrated services.}
      \resumeItem {Worked with Oracle databases, wrote extensive SQL queries and stored procedures, and worked on the improvement of Java Spring APIs via best practices.}
      \resumeItem {Created CI/CD pipelines on Azure DevOps for seamless deployment/execution of the migrated services.}
    \resumeItemListEnd
  \resumeEntryEnd

  \resumeEntryStart
    \resumeEntryTSDL
      {Johnson Controls}{January 2022 -- July 2022}
      {Software Developer Intern}{Cork, Ireland}
    \resumeItemListStart
      \resumeItem {Collaborated on the design and development of an enterprise-grade chatbot using Node.js and Azure Bot Framework SDK, hosted on Microsoft Azure Cloud.}
      \resumeItem {Implemented the core web service of the chatbot that communicated with both internal and external REST API microservices.}
      \resumeItem {Utilized natural language processing AI models available on Azure Cognitive Services to enhance the chatbot's ability to understand and respond to user input.}
      \resumeItem {Worked within an agile development process utilizing Azure DevOps and Git version control.}
    \resumeItemListEnd
  \resumeEntryEnd

  \resumeEntryStart
    \resumeEntryTSDL
      {Adastec}{June 2021 -- August 2021}
      {Software Engineer Intern}{Istanbul, Turkey}
    \resumeItemListStart
      \resumeItem {Worked in the perception part of the autonomous driving software for commercial vehicles.}
      \resumeItem {Designed and developed a service using C++ to detect the position and the state (opened or closed) of access barriers using 3D space LIDAR sensor data.}
      \resumeItem {Utilized pub/sub communication with other services to ensure the raw LIDAR sensor data was compressed, noiseless, and semanticized.}
    \resumeItemListEnd
  \resumeEntryEnd


%-------------------------------------------------- PROJECTS --------------------------------------------------
\section{\faFlask}{Personal Projects}

  \resumeEntryStart
    \resumeEntryTD
      {CLOWA \href{https://clowa.net}{(clowa.net)}}{\githubicon \href{https://github.com/Roi-des-Rats/clowa}{ /clowa}}
    \resumeItemListStart
      \resumeItem {CLOWA (Curated List Of Web Articles) is a website where a handful of curators share articles, blog posts, and other written web content that they think are worth reading. Written using Next.js, Tailwind CSS, and TypeScript with a Supabase backend (PostgreSQL).}
    \resumeItemListEnd
  \resumeEntryEnd

  \resumeEntryStart
    \resumeEntryTD
      {Claw.js}{\githubicon \href{https://github.com/yvesyil/claw-js}{ /claw-js}}
    \resumeItemListStart
      \resumeItem {A linear algebra library for JavaScript that's written in both C and JavaScript that uses OpenCL to compute matrices on the GPU. It's aimed to be used for Deep Learning applications as it can reduce training time.}
    \resumeItemListEnd
  \resumeEntryEnd

  \resumeEntryStart
    \resumeEntryTD
      {Neural Network from Scratch}{\githubicon \href{https://github.com/yvesyil/neural-net-api}{ /neural-net-api}}
    \resumeItemListStart
      \resumeItem {A Neural network completely written from scratch using TypeScript with matrix and differential operations are written by me, designed to recognize hand-written digits. Also serves as a REST API.}
    \resumeItemListEnd
  \resumeEntryEnd

%-------------------------------------------------- PROGRAMMING SKILLS --------------------------------------------------
\section{\faGears}{Skills}
  \resumeEntryStart
    \resumeEntryS{Programming Languages} {JavaScript, TypeScript, Python, Java, Go, C, C++, HTML, CSS, SQL}
    \resumeEntryS{Tools \& Technologies} {Git, Node.js, Oracle, PostgreSQL, Docker, Kubernetes, Microsoft Azure, Azure DevOps, Linux, CMake, Ansible, Coreutils \& binutils}
    \resumeEntryS{Frameworks \& Libraries} {React, Express.js, Next.js, Flask, Pandas, Numpy, PyTorch, OpenGL, OpenCL, WebGL, Tailwind}
    \resumeEntryS{Natural Languages} {Turkish (Native), English (Advanced), French (Basic), Dutch (Basic)}
  \resumeEntryEnd

%-------------------------------------------------- Hobbies --------------------------------------------------
\section{\faHeart}{Hobbies}
  \resumeEntryStart
    \resumeEntryTD{Music}{}
    \resumeItemListStart
      \resumeItem{Playing Guitar and Piano}
      \resumeItem{Music Composition and Theory}
    \resumeItemListEnd
    \resumeEntryTD{Reading}{}
    \resumeItemListStart
      \resumeItem{Technical Blogs}
      \resumeItem{Materials related to STEM}
    \resumeItemListEnd
    \resumeEntryTD{Sports}{}
    \resumeItemListStart
      \resumeItem{Bouldering}
      \resumeItem{Ping-Pong}
    \resumeItemListEnd
  \resumeEntryEnd

\end{document}